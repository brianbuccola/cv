%==============================================================================%
%                                                                              %
%  file:   buccola-cv.tex                                                      %
%  author: Brian Buccola <brian.buccola@gmail.com>                             %
%                                                                              %
%==============================================================================%

%============%
%            %
%  PREAMBLE  %
%            %
%============%

\documentclass[11pt,a4paper,twoside,french]{article}

%===============================%
%  Packages and basic settings  %
%===============================%

\newcommand{\name}{Brian Buccola}

% Font settings
\usepackage[T1]{fontenc}
\usepackage[utf8]{inputenc}
\usepackage{microtype}
\usepackage[sc,osf]{mathpazo}% Palatino font, w/ smallcaps & old-style figures
\linespread{1.05}
\usepackage{marvosym}% provides \Letter and \Telefon

% French
\usepackage{babel}
\usepackage{csquotes}

% Margins
\usepackage[margin=1.5in,twoside]{geometry}

% Section fonts
\usepackage{titlesec}
\titleformat*{\section}{\normalfont\Large\scshape}
\titleformat*{\subsection}{\normalfont\large\bfseries}
\titleformat*{\subsubsection}{\normalfont\normalsize\itshape}

% Headers, footers, and page styles
\usepackage{titleps}
\newpagestyle{main}[\footnotesize]{%
  \headrule
  \sethead[Curriculum Vit\ae][][\name]
          {Curriculum Vit\ae}{}{\name}
  \setfoot[][\thepage][\scriptsize \today]
          {}{\thepage}{\scriptsize \today}
}
\newpagestyle{first}[\footnotesize]{%
  \sethead{}{}{}
  \setfoot{}{}{\scriptsize \today}
}
\pagestyle{main}

% References
\usepackage[backend=biber,style=numeric,maxnames=99]{biblatex}
\addbibresource{buccola-cv.bib}
\renewcommand{\subtitlepunct}{\addcolon\space}
%% Remove default quotes around title.
\DeclareFieldFormat[article,inproceedings,unpublished]{citetitle}{#1}
\DeclareFieldFormat[article,inproceedings,unpublished]{title}{#1}
\DeclareFieldFormat[thesis]{citetitle}{\emph{#1}}
\DeclareFieldFormat[thesis]{title}{\emph{#1}}
%% Don't print pubstate field.
\AtEveryCitekey{\clearfield{pubstate}}

% Packages for CV environment
\usepackage{array}% required for defining newcolumntype with custom vrule
\usepackage{longtable}% normal \tabular environment does not allow page breaks
\setlength{\LTpre}{0pt}% glue before longtable
\addtolength{\LTpost}{0pt}% glue after longtable

% Hyper links and PDF info
\usepackage{hyperref}
\hypersetup{%
  colorlinks  = false,
  pdfauthor   = {\name},
  pdftitle    = {Curriculum Vitae -- \name},
  pdfsubject  = {Curriculum Vitae},
  pdfkeywords = {\name, CV, linguistique},
  pdfpagemode = UseNone
}

\setlength{\parindent}{0pt}

%==========%
%  Macros  %
%==========%

\newcommand{\datewidth}{0.18}
\newcommand{\bodywidth}{0.78}

\newcolumntype{L}{>{\raggedright}p{\datewidth\textwidth}}
\newcolumntype{R}{p{\bodywidth\textwidth}}

\newenvironment{cvsection}{%
  \setlength{\extrarowheight}{0.70ex}
  \begin{longtable}[l]{@{} L R @{}}
  % Comment line above and uncomment line below to add gray vrule between date and body.
  % \begin{longtable}{@{} L !{\myvrule} R @{}}
}{%
  \end{longtable}
}

\newcommand{\taship}[3]{% list of TA course, duties, and instructor
  \parbox[t]{\bodywidth\textwidth}{#1.\\ {\footnotesize Fonctions : #2.\\
      Chargé de cours : #3.}}
}

\newcommand{\course}[3]{% course name, course description, and class size
  \parbox[t]{\bodywidth\textwidth}{#1.\\ {\footnotesize Description du cours : #2.
      Nombre d'élèves : #3.}}
}

\newcommand{\award}[2]{%
  #1 (#2). % award description and award amount
}

%========%
%        %
%  BODY  %
%        %
%========%

\begin{document}

\thispagestyle{first}

\begin{center}
  {\Huge\bfseries \name}
\end{center}

\vspace{1em}

\section*{Coordonées}

{\footnotesize
  \begin{minipage}[t]{0.58\textwidth}
    Laboratoire de Sciences Cognitives et Psycholinguistique\\
    École Normale Supérieure\\
    29 rue d'Ulm, 75005 Paris\\
    France
  \end{minipage}
  \hfill
  \begin{minipage}[t]{0.32\textwidth}
    \Telefon\ +1 312-945-6857 {\footnotesize (É.-U.)}\\
    \Telefon\ +33 7 83 88 22 19 {\footnotesize (France)}\\
    \Letter\ \href{mailto:brian.buccola@gmail.com}{\ttfamily brian.buccola@gmail.com}\\
    \Keyboard\ \href{http://brianbuccola.com/}{\ttfamily brianbuccola.com}
  \end{minipage}
}

\section*{Expérience professionnelle}

\begin{cvsection}
  2017-- & \parbox[t]{\bodywidth\textwidth}{%
    Chercheur postdoctoral, Laboratoire de Sciences Cognitives et Psycholinguistique (ENS, EHESS, CNRS), Département d'Études Cognitives, École Normale Supérieure, PSL Research University.\\
    {\footnotesize PI : Emmanuel Chemla}
  }\\
  \parbox[t]{\datewidth\textwidth}{%
    2016\\
    {\footnotesize (juin 25 -- juil.\ 27)}
    } & \parbox[t]{\bodywidth\textwidth}{%
    Chercheur invité, Institut Jean Nicod (ENS, EHESS, CNRS), Département d'Études Cognitives, École Normale Supérieure, PSL Research University.\\
    {\footnotesize Sponsor : Benjamin Spector}
  }\\
  2015--2017 & \parbox[t]{\bodywidth\textwidth}{%
    Chercheur postdoctoral, Language, Logic and Cognition Center, The Hebrew University of Jerusalem.\\
    {\footnotesize PI : Luka Crnič et Yosef Grodzinsky}
  }\\
\end{cvsection}

\section*{Éducation}

\begin{cvsection}
  2010--2015 & \parbox[t]{\bodywidth\textwidth}{%
    Ph.D., Linguistique, Université McGill.\\
    {\footnotesize Thèse : \emph{Maximality in the Semantics of Modified Numerals}}\\
    {\footnotesize Comité de thèse : Luis Alonso-Ovalle (co-directeur), Bernhard Schwarz (co-directeur), Michael Wagner (membre)}
  }\\
  2014 & \parbox[t]{\bodywidth\textwidth}{%
    Étudiant visiteur, Massachusetts Institute of Technology.\\
    {\footnotesize Sponsor : Kai von Fintel}
  }\\
  2009--2010 & \parbox[t]{\bodywidth\textwidth}{%
    Qualifying Year Program, Université McGill.\\
    {\footnotesize (Permet une mise à niveau dans les disciplines cruciales pour les bons candidats au programme doctoral.)}
  }\\
  2004--2008 & \parbox[t]{\bodywidth\textwidth}{%
    B.S., Mathématiques, Honors College, Loyola University Chicago.\\
    B.A., Lettres classiques, Honors College, Loyola University Chicago.
  }\\
  1999--2004 & Jesuit High School New Orleans.\\
\end{cvsection}

\subsection*{Universités d'été}

\begin{cvsection}
  2013 & European Summer School in Logic, Language, and Information (ESSLLI 2013), D\"{u}sseldorf, Allemagne.\\
  2012 & North American Summer School of Logic, Language, and Information (NASSLLI 2012), Austin, TX.\\
\end{cvsection}

\section*{Recherche}

\subsection*{Articles dans des revues avec comité de lecture}

\begin{cvsection}
  2016 & \fullcite{buccola.spector2016}.\\
  2012 & \fullcite{schwarz.etal2012}.\\
\end{cvsection}

\subsection*{Manuscrits}

\begin{cvsection}
  soumis (snippet) & \fullcite{buccola2018only}.\\
  soumis (snippet) & \fullcite{buccola.chemla2018or}.\\
  soumis & \fullcite{buccola.etal2018}.\\
  soumis & \fullcite{chemla.etal2018}.\\
  en révision & \fullcite{buccola.haida2017oi}.\\
  en révision & \fullcite{buccola2017barenum}.\\
\end{cvsection}

\subsection*{Articles dans des actes de conférence avec comité de lecture}

\begin{cvsection}
  2017 & \fullcite{buccola.haida2017ac}.\\
  2017 & \fullcite{buccola.haida2017sub}.\\
  2016 & \fullcite{buccola2016sub}.\\
  2015 & \fullcite{buccola2015nels}.\\
  2013 & \fullcite{buccola2013fg}.\\
\end{cvsection}

\subsection*{Présentations}

\subsubsection*{Conférences avec comité de lecture}

\begin{cvsection}
  2017 & \fullcite{buccola.haida2017actalk}.\\
  2016 & \fullcite{buccola2016logicon}.\\
  2016 & \fullcite{buccola.haida2016logicon}.\\
  2016 & \fullcite{buccola.haida2016subtalk}.\\
  2016 & \fullcite{buccola2016uab}.\\
  2015 & \fullcite{buccola2015subtalk}.\\
  2015 & \fullcite{buccola2015sle}.\\
  2014 & \fullcite{buccola2014nelsposter}.\\
  2014 & \fullcite{buccola.sonderegger2014lsa}.\\
  2013 & \fullcite{buccola.sonderegger2013phon}.\\
  2013 & \fullcite{buccola2013fgtalk}.\\
\end{cvsection}

\subsubsection*{Ateliers}

\begin{cvsection}
  2016 & \fullcite{buccola.haida2016eingedi}.\\
  2014 & \fullcite{buccola2014tom}.\\
  2013 & \fullcite{buccola2013mot}.\\
  2012 & \fullcite{buccola2012tom}.\\
  2011 & \fullcite{buccola2011tom}.\\
\end{cvsection}

\subsection*{Présentations invitées}

\begin{cvsection}
  2017 & \fullcite{buccola2017linguae}.\\
  2017 & \fullcite{buccola2017huji}.\\
  2016 & \fullcite{buccola2016linguae}.\\
  2016 & \fullcite{buccola2016llcc}.\\
  2015 & \fullcite{buccola2015tau}.\\
  2015 & \fullcite{buccola2015lush}.\\
\end{cvsection}

\subsection*{Éditions d'actes de conférences}

\begin{cvsection}
  2014 & \fullcite{buccola.etal2014}.\\
  2013 & \fullcite{mckillen.buccola2013}.\\
\end{cvsection}

\subsection*{Thèse}

\begin{cvsection}
  2015 & \fullcite{buccola2015diss}. Comité de thèse : Luis Alonso-Ovalle (co-directeur), Bernhard Schwarz (co-directeur), Michael Wagner (membre).\\
\end{cvsection}

\section*{Enseignement}

\subsection*{Comme chargé de cours}

\begin{cvsection}
  Été 2014 & \course{Introduction to Linguistics (LING 201), Université McGill}
  {introduction aux techniques de l'analyse linguistique; sujets couverts : la phonétique, la phonologie, la morphologie, la syntaxe, et la sémantique}
  {30}
\end{cvsection}

\subsection*{Comme assistant}

\begin{cvsection}
  Hiver 2014 & \taship{Phonetics (LING 330), Université McGill}
  {conférence hebdomadaire (2 heures/semaine); heures de permanence; noter les devoirs et les examens}
  {Morgan Sonderegger}\\

  Automne 2013 & \taship{Semantics (LING 360), Université McGill}
  {trois conférences; heures de permanence; noter les devoirs et les examens}
  {Brendan Gillon}\\

  Hiver 2013 & \taship{Phonetics (LING 330), Université McGill}
  {conférence hebdomadaire (2 heures/semaine); heures de permanence; noter les devoirs et les examens}
  {Morgan Sonderegger}\\

  Automne 2012 & \taship{Semantics (LING 360), Université McGill}
  {deux conférences; heures de permanence; noter les devoirs et les examens}
  {Brendan Gillon}\\

  Hiver 2012 & \taship{Phonetics (LING 330), Université McGill}
  {conférence hebdomadaire (2 heures/semaine); heures de permanence; noter les devoirs et les examens}
  {Heather Goad}\\

  Automne 2011 & \taship{Phonology I (LING 331), Université McGill}
  {conférence hebdomadaire (1 hour/semaine); heures de permanence; noter les tests, les devoirs, et les examens}
  {Tobin Skinner}\\

  Hiver 2011 & \taship{Phonetics (LING 330), Université McGill}
  {conférence hebdomadaire (2 heures/semaine); heures de permanence; noter les tests, les devoirs, et les examens}
  {Sara Mackenzie}\\

  Automne 2010 & \taship{Introduction to Linguistics (LING 201), Université McGill}
  {conférence hebdomadaire (2 heures/semaine); heures de permanence; noter les tests, les devoirs, et les examens}
  {Tanya Slavin}\\

  Automne 2010 & \taship{Neuroscience of Language (LING 370), Université McGill}
  {noter les devoirs et les examens}
  {Isabelle Deschamps, Andrea Santi}\\
\end{cvsection}

\section*{Autres contributions}

\subsection*{Au département}

\begin{cvsection}
  2015--2017 & Organisateur (avec Andreas Haida), LLCC Semantics-Pragmatics Research Group, The Hebrew University of Jerusalem.\\
  2015 & \parbox[t]{\bodywidth\textwidth}{%
    Department publications manager, Université McGill.\\
    {\footnotesize Compiler et gérer une base de données des publications du département pour 2014, au format Bib\TeX, pour l'Annual Report du département.}
  }\\
  2013--2015 & Comité d'organisation, Syntax-Semantics Research Group, Université McGill.\\
  2012--2014 & Co-éditeur, McGill Working Papers in Linguistics (McGWPL).\\
  2013 & Comité d'organisation, Exploring the Interfaces 2: Implicatures, Alternatives, and the Semantics-Pragmatics Interface (ETI~2), Université McGill.\\
  2012 & Représantant-étudiant pour le poste en syntaxe, Université McGill.\\
  2011--2012 & Co-éditeur, McGill Linguistics Blog (McLing).\\
\end{cvsection}

\subsection*{À la profession}

\begin{cvsection}
  2016-- & Membre de l'Editorial Board de \emph{Semantics and Pragmatics}.\\
  Reviewer (livres) & The MIT Press (1 manuscrit, 2016).\\
  Reviewer (revues) & \emph{Acta Linguistica Academica} (1 manuscrit, 2017), \emph{Glossa} (1 manuscrit, 2017), \emph{Journal of Semantics} (2 manuscrits, 2015--2016), \emph{Semantics and Pragmatics} (2 manuscrits, 2016--2017).\\
  Reviewer (conférences) & \emph{GLOW in Asia} (2016), \emph{Israel Association for Theoretical Linguistics} (2016--2017), \emph{North East Linguistic Society} (2015--2017), \emph{Semantics and Linguistic Theory} (2018), \emph{Sinn und Bedeutung} (2017).
\end{cvsection}

\section*{Prix}

\begin{cvsection}
  2014 & \parbox[t]{\bodywidth\textwidth}{%
    \award{Faculty of Arts Graduate Student Teaching Award}{\$500}\\
    {\footnotesize Prix sur nomination : ``recognizes outstanding teaching in the Faculty by graduate students.''}
  }\\
  2014 & \parbox[t]{\bodywidth\textwidth}{%
    \award{Centre for Research on Brain, Language and Music (CRBLM) Graduate Travel Award}{\$700}\\
    {\footnotesize Prix sur demande octroyé aux membres-étudiants du CRBLM pour présenter leurs résultats scientifiques à une conférence internationale.}
  }\\
  2013 & \parbox[t]{\bodywidth\textwidth}{%
    \award{Graduate Student Travel Award, Université McGill}{\$1,500}\\
    {\footnotesize Un ``competitive award designed to support graduate student travel by both MA and PhD students for the purposes of research and the dissemination of research.''}
  }\\
  2013 & \parbox[t]{\bodywidth\textwidth}{%
    \award{Lara Riente Memorial Prize in Linguistics, Department of Linguistics, Université McGill}{\$500}\\
    {\footnotesize ``Awarded by the Faculty of Arts Scholarships committee on the recommendation of the Department of Linguistics on the basis of high academic standing to a graduate or an undergraduate student enrolled in a full-time degree program in Linguistics.''}
  }\\
  2012--2013 & \parbox[t]{\bodywidth\textwidth}{%
    \award{Cremona Memorial Fellowship in Linguistics, Department of Linguistics, Université McGill}{\$8,479}\\
    {\footnotesize ``Awarded by the Department of Linguistics to an outstanding graduate student registered in the Ph.D.\ or M.A.\ program in Linguistics.''}
  }\\
  2012 & \parbox[t]{\bodywidth\textwidth}{%
    \award{NASSLLI full scholarship}{\$410}\\
    {\footnotesize Bourse pour couvrir les frais, le logement, et le transport.}
  }\\
  2004--2008 & \parbox[t]{\bodywidth\textwidth}{%
    \award{National Merit Finalist Scholarship, Loyola University Chicago}{\textasciitilde\$100,000}\\
    {\footnotesize Bourse ``full-tuition'' octroyée aux étudiants qui ont été nommés National Merit Finalists par la National Merit Corporation.}
  }\\
\end{cvsection}

\section*{Distinctions et associations}

\begin{cvsection}
  2016 {\footnotesize (inscrit)} & Israel Association for Theoretical Linguistics (IATL).\\
  2015 {\footnotesize (inscrit)} & Societas Linguistica Europaea (SLE).\\
  2012 {\footnotesize (inscrit)} & Centre for Research on Brain, Language and Music (CRBLM), Montr\'{e}al, Canada.\\
  2012 {\footnotesize (inscrit)} & Linguistic Society of America (LSA).\\
  2008 {\footnotesize (initié)} & Phi Beta Kappa.\\
  2004 & National Merit Scholar.\\
  1999 {\footnotesize (accepté)} & Mensa.\\
\end{cvsection}

\section*{Langues}

\begin{cvsection}
  Anglais & Langue maternelle\\
  Italien & Avancé\\
  Français, Allemand & Intermédiaire\\
  Latin, Grec & Compétent (traduction, recherche)\\
  Espagnol & Connaissance pratique (recherche)\\
  Hébreu & Débutant\\
\end{cvsection}

\section*{Compétence technique}

\begin{cvsection}
  Programmation & Linux shell (Bash, Zsh); Haskell, Python, R; Git; \LaTeX, Bib\TeX; HTML, CSS\\
  Logiciel pour la linguistique & Ibex; Praat\\
\end{cvsection}

% \section*{Autres passions}

% \begin{list}{}{\leftmargin=0pt}
%   \item L'histoire et la philosophie des mathématiques et de la science.
%   \item Le blues (Delta) et le jazz (de la Nouvelle-Orléans).
%   \item Linux, \LaTeX{}, l'informatique.
%   \item Le brassage de la bière, la cuisine.
%   \item Échecs, les casse-têtes.
% \end{list}

\end{document}
