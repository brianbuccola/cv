%===============================================================================
%
%  vim:     set fenc=utf8 tw=79 sw=2 sts=2 expandtab:
%  file:    buccola-cv.tex
%  author:  Brian Buccola
%
%===============================================================================



%============%
%            %
%  PREAMBLE  %
%            %
%============%

\documentclass[letterpaper]{article}



%=============================
% Packages and basic settings
%=============================

\newcommand{\name}{Brian Buccola}

\usepackage[T1]{fontenc}
\usepackage[sc,osf]{mathpazo}

\usepackage[dvipsnames]{xcolor}
\newcommand{\linkcolor}{Blue}

\usepackage{array} % required for defining newcolumntype with custom vrule

\usepackage{longtable} % normal \tabular environment does not allow page breaks

\usepackage[margin=1in]{geometry}

\usepackage{needspace}

\usepackage{sectsty}
\sectionfont{\needspace{3\baselineskip} \rmfamily\mdseries\scshape\large \sectionrule{0pt}{0pt}{-1ex}{1pt}}
\subsectionfont{\needspace{2\baselineskip} \rmfamily\mdseries\scshape\normalsize}

\usepackage{natbib}
\usepackage{bibentry}

\usepackage{datetime}
\newdateformat{mydate}{\twodigit{\THEDAY}\ \shortmonthname[\THEMONTH] \THEYEAR}
\mydate

\newdateformat{mydatenoday}{\shortmonthname[\THEMONTH]\ \THEYEAR}
\newcommand{\formatdatenoday}[2]{\mydatenoday\formatdate{0}{#1}{#2}}

\usepackage{hyperref}
\hypersetup{%
  colorlinks  = true,
  allcolors   = \linkcolor,
  pdfauthor   = {\name},
  pdftitle    = {Curriculum Vitae -- \name},
  pdfsubject  = {Curriculum Vitae},
  pdfkeywords = {\name, CV, linguistics, McGill},
  pdfpagemode = UseNone
}

\hyphenation{Chic-ago}
\hyphenation{work-shop}

\setlength\parindent{0pt}

\thispagestyle{plain}
\pagestyle{plain}

\setlength{\LTpre}{0pt} % glue before longtable
\addtolength{\LTpost}{2ex} % glue after longtable



%========
% Macros
%========

\newcommand{\datewidth}{0.16}
\newcommand{\bodywidth}{0.81}

\definecolor{lightgray}{gray}{0.85}

\newcommand{\myvrule}{\color{lightgray}\vrule width 1.0pt}

\newcolumntype{L}{>{\raggedright}p{\datewidth\textwidth}}
\newcolumntype{R}{p{\bodywidth\textwidth}}

\newenvironment{cvsection}{%
  \renewcommand{\arraystretch}{1.75}
  \begin{longtable}[l]{@{} L R @{}}
  % Uncomment line below to add gray vrule between date and body.
  % \begin{longtable}{@{} L !{\myvrule} R @{}}
}{%
  \end{longtable}
}

\newcommand{\taship}[3]{% list of TA course, duties, and instructor
  \parbox[t]{\bodywidth\textwidth}{#1. \\ {\footnotesize Duties included: #2.
  \\ Instructor: #3.}}
}

\newcommand{\course}[2]{% course name and course description
  \parbox[t]{\bodywidth\textwidth}{#1. \\ {\footnotesize Course description:
      #2.}}
}

\newcommand{\award}[2]{%
  #1~(#2). % award description and award amount
}



%========%
%        %
%  BODY  %
%        %
%========%

\begin{document}

\bibliographystyle{plain}
\nobibliography{/home/brian/documents/references/references}



%=========================================
% Title, personal info, research interests
%=========================================

\begin{center}
  {\Huge\scshape \name} \\[\baselineskip]

  Curriculum Vit\ae \\
  \today
\end{center}



\section*{Personal Information}

\begin{cvsection}
  Born & On \formatdate{29}{12}{1985}, in Metairie, Louisiana \\

  Citizenship & USA \\

  Address & \parbox[t][5\baselineskip]{\bodywidth\textwidth}{%
    \href{http://www.mcgill.ca/linguistics/}{Department of Linguistics} \\
    \href{http://www.mcgill.ca/}{McGill University} \\
    1085 Dr-Penfield \\
    Montr\'{e}al, Qu\'{e}bec H3A 1A7 \\
    Canada
  } \\

  Email & \href{mailto:brian.buccola@mail.mcgill.ca} {\ttfamily
  brian.buccola@mail.mcgill.ca} \\

  Website & \url{http://people.linguistics.mcgill.ca/~brian.buccola/}
\end{cvsection}



\section*{Research Interests}

\begin{cvsection}
  General & Formal semantics, pragmatics; phonology; mathematical and
  computational linguistics; philosophy of language. \\

  Semantics & Quantification; numeral and scalar modification; plurality;
  distributivity\slash collectivity; nonmonotonic inferences, ignorance
  inferences; type logical grammar. \\

  Phonology & Opacity; Optimality Theory; finite state phonology; generative
  capacity of phonological grammars. \\
\end{cvsection}



%===========
% Main body
%===========

\section*{Education}

\begin{cvsection}
  2010-- & Ph.D.~candidate, Linguistics, \href{http://www.mcgill.ca/}{McGill
  University}. \\

  2009--2010 & Qualifying year, Linguistics,
  \href{http://www.mcgill.ca/}{McGill University}. \\

  2004--2008 & B.S.~Mathematics, Honors College,
  \href{http://www.luc.edu/}{Loyola University Chicago}. \\

  2004--2008 & B.A.~Classics, Honors College, \href{http://www.luc.edu/}{Loyola
  University Chicago}. \\

  1999--2004 & \href{http://www.jesuitnola.org/about/aboutindex.htm}{Jesuit
  High School New Orleans}. \\
\end{cvsection}

\vspace{-1ex}
\subsection*{Summer Schools}

\begin{cvsection}
  2013 & European Summer School in Logic, Language, and Information
  (\href{http://esslli2013.de/}{ESSLLI 2013}), D\"{u}sseldorf, Germany,
  \formatdatenoday{8}{2013}. \\

  2012 & North American Summer School of Logic, Language, and Information
  (\href{http://nasslli2012.com/}{NASSLLI 2012}), Austin, TX,
  \formatdatenoday{6}{2012}. \\
\end{cvsection}



\section*{Refereed Papers}

\begin{cvsection}
  2013 & \null\bibentry{buccola2013fg}.
  [\href{http://people.linguistics.mcgill.ca/~brian.buccola/files/buccola2013-fg.pdf}{preprint}] \\

  2012 & \null\bibentry{schwarz.etal2012}.
  [\href{http://dx.doi.org/10.3765/sp}{doi}] \\
\end{cvsection}



\section*{Evaluation Papers}

\begin{cvsection}
  2013 & \null\bibentry{buccola2013eval2}.
  [\href{http://people.linguistics.mcgill.ca/~brian.buccola/files/buccola2013-eval2.pdf}{draft}] \\

  2012 & \null\bibentry{buccola2012eval1}.
  [\href{http://people.linguistics.mcgill.ca/~brian.buccola/files/buccola2012-eval1.pdf}{draft}] \\
\end{cvsection}



\section*{Refereed Presentations}

\begin{cvsection}
  2014 & \null\bibentry{buccola.sonderegger2014lsa}. \\
  2013 & \null\bibentry{buccola.sonderegger2013phon}. \\
  2013 & \null\bibentry{buccola2013fgtalk}. \\
  2013 & \null\bibentry{buccola2013mot}.
  [\href{http://people.linguistics.mcgill.ca/~brian.buccola/files/buccola2013-mot.pdf}{handout}] \\

  2012 & \null\bibentry{buccola2012tom}.
  [\href{http://people.linguistics.mcgill.ca/~brian.buccola/files/buccola2012-tom5.pdf}{handout}] \\

  2011 & \null\bibentry{buccola2011tom}.
  [\href{http://people.linguistics.mcgill.ca/~brian.buccola/files/buccola2011-tom4.pdf}{handout}] \\
\end{cvsection}



\section*{Non-Refereed Presentations}

\begin{cvsection}
  2013 & \name. Scales and scalar modifiers. Syntax-Semantics Research Group,
  McGill University, \formatdate{13}{9}{2013}. \\

  2012 & \name. Formalization in linguistics.  Invited lecture, Brendan
  Gillon's pragmatics course (LING 565), McGill University,
  \formatdate{5}{11}{2012}. \\

  2012 & \name. Some remarks on inference patterns involving epistemic
  modality, part 2. Syntax-Semantics Research Group, McGill University,
  \formatdate{31}{7}{2012}. \\

  2012 & \name. Some remarks on inference patterns involving epistemic
  modality, part 1. Syntax-Semantics Research Group, McGill University,
  \formatdate{11}{7}{2012}. \\

  2012 & \name. On Cummins and Katsos (2010): Comparative and superlative
  quantifiers: Pragmatic effects of comparison type. Syntax-Semantics Research
  Group, McGill University, \formatdate{25}{11}{2012}. \\

  2011 & \name, Michael Hamilton, and Bernhard Schwarz. On Nouwen (2010): Two
  kinds of modified numerals, part 2. Syntax-Semantics Research Group, McGill
  University, \formatdate{22}{6}{2011}. \\

  2011 & \name, Michael Hamilton, and Bernhard Schwarz. On Nouwen (2010): Two
  kinds of modified numerals, part 1. Syntax-Semantics Research Group, McGill
  University, \formatdate{8}{6}{2011}. \\
\end{cvsection}



\section*{Edited Volumes}

\begin{cvsection}
  2014 & \null\bibentry{buccola.etal2014}.
  [\href{https://www.mcgill.ca/mcgwpl/archives/volume-241-2014}{issue}] \\
  2013 & \null\bibentry{mckillen.buccola2013}.
  [\href{https://www.mcgill.ca/mcgwpl/archives/volume-231-2013}{issue}]
\end{cvsection}



\section*{Awards}

\begin{cvsection}
  2014 & \award{Faculty of Arts Graduate Student Teaching Award}{\$500} \\

  2014 & \award{Centre for Research on Brain, Language and Music (CRBLM)
  Graduate Travel Award}{\$700} \\

  2013 & \award{Graduate Student Travel Award, McGill University}{\$1,500} \\

  2013 & \award{Lara Riente Memorial Prize in Linguistics, Department of
  Linguistics, McGill University}{\$500} \\

  2012--2013 & \award{Cremona Memorial Fellowship in Linguistics, Department of
  Linguistics, McGill University}{\$8,479} \\

  2012--2013 & \award{Graduate Excellence Fellowship, McGill
  University}{\$7,000} \\

  2012 & \award{NASSLLI full scholarship (tuition, fees, accommodation, travel
  stipend)}{\$410} \\

  2011--2012 & \award{Graduate Excellence Fellowship, McGill
  University}{\$6,500} \\

  2010--2011 & \award{Graduate Enrolment Recruitment Initiative, McGill
  University}{\$5,000} \\

  2010--2011 & \award{McCall MacBain Award, McGill University}{\$10,000} \\

  2007--2008 & \award{National SMART Grant (Science and Mathematics Access to
  Retain Talent), Loyola University Chicago}{\$4,000} \\

  2004--2008 & \award{National Merit Finalist Scholarship, Loyola University
  Chicago}{\textasciitilde\$100,000} \\
\end{cvsection}



\section*{Honors and Associations}

\begin{cvsection}
  2012 {\footnotesize (joined)} & \href{http://www.crblm.ca/}{Centre for
  Research on Brain, Language and Music} (CRBLM, Montr\'{e}al, Canada). \\

  2012 {\footnotesize (joined)} &
  \href{http://www.linguisticsociety.org/}{Linguistic Society of America}. \\

  2008 {\footnotesize (inducted)} & \href{http://www.pbk.org/}{Phi Beta Kappa}. \\

  2004 & \href{http://www.nationalmerit.org/}{National Merit} Scholar. \\

  1999 {\footnotesize (accepted)} & \href{http://www.mensa.org/}{Mensa}. \\
\end{cvsection}



\section*{Teaching}

\subsection*{As Instructor}

\begin{cvsection}
  {\small Summer} 2014 & \course{Introduction to Linguistics (LING 201),
    instructor, McGill University}{introduction to the techniques of linguistic
    analysis; topics covered include phonetics, phonology, morphology, syntax,
    and semantics.}
  \\[0.10ex]
\end{cvsection}

\vspace{-1ex}
\subsection*{As Teaching Assistant}

\begin{cvsection}
  {\small Winter} 2014 & \taship{Phonetics (LING 330), teaching assistant,
  McGill University}{weekly lecture (2 hours/week); office hours; grade
  assignments, midterm exam, and final
  exam}{\href{http://people.linguistics.mcgill.ca/~morgan/}{Morgan Sonderegger}}
  \\[0.10ex]

  {\small Fall} 2013 & \taship{Semantics (LING 360), teaching assistant, McGill
  University}{three lectures; office hours; grade assignments, midterm exam, and
  final exam}{\href{http://webpages.mcgill.ca/staff/group3/bgillo/web/}{Brendan
  Gillon}}
  \\[0.10ex]

  {\small Winter} 2013 & \taship{Phonetics (LING 330), teaching
  assistant, McGill University}{weekly lecture (2 hours/week); office hours;
  grade assignments, midterm exam, and final
  exam}{\href{http://people.linguistics.mcgill.ca/~morgan/}{Morgan Sonderegger}}
  \\[0.10ex]

  {\small Fall} 2012 & \taship{Semantics (LING 360), teaching assistant,
  McGill University}{two lectures; office hours; grade assignments, midterm
  exam, and final
  exam}{\href{http://webpages.mcgill.ca/staff/group3/bgillo/web/}{Brendan
  Gillon}}
  \\[0.10ex]

  {\small Winter} 2012 & \taship{Phonetics (LING 330), teaching
  assistant, McGill University}{weekly lecture (2 hours/week); office hours;
  grade assignments, midterm exam, and final
  exam}{\href{http://webpages.mcgill.ca/staff/group3/hgoad/web/}{Heather Goad}}
  \\[0.10ex]

  {\small Fall} 2011 & \taship{Phonology I (LING 331), teaching
  assistant, McGill University}{weekly lecture (1 hour/week); office hours;
  grade quizzes, assignments, midterm exam, and final
  exam}{\href{http://tobinskinner.com/}{Tobin Skinner}}
  \\[0.10ex]

  {\small Winter} 2011 & \taship{Phonetics (LING 330), teaching assistant,
  McGill University}{weekly lecture (2 hours/week); office hours; grade
  quizzes, assignments, midterm exams, and final
  exam}{\href{http://www.mun.ca/linguistics/people/faculty/saramackenzie.php}{Sara
  Mackenzie}}
  \\[0.10ex]

  {\small Fall} 2010 & \taship{Intro to Linguistics (LING 201), teaching
  assistant, McGill University}{weekly lecture (2 hours/week); office hours;
  grade quizzes, assignments, midterm exam, and final
  exam}{\href{https://sites.google.com/site/tanyaslavin/}{Tanya Slavin}}
  \\[0.10ex]

  {\small Fall} 2010 & \taship{Neuroscience of Language (LING 370),
  grader, McGill University}{grade assignments, midterm exam, and final
  exam}{Isabelle Deschamps, Andrea Santi} \\
\end{cvsection}



\section*{Service}

\begin{cvsection}
  2013-- & Organizing committee, Syntax-Semantics Research Group, McGill
  University. \\

  2013 & Organizing committee, Exploring the Interfaces 2: Implicatures,
  Alternatives, and the Semantics-Pragmatics Interface
  (\href{https://sites.google.com/site/eti2pragmatics/}{ETI~2}), McGill
  University, \shortmonthname[4]--\formatdatenoday{6}{2013}. \\

  2013 & Session chair, Toronto-Ottawa-Montreal (TOM) semantics workshop,
  McGill University, \formatdate{23}{3}{2013}. \\

  2013 & Student host for invited colloquium speaker Kai von Fintel, McGill
  University, \formatdatenoday{1}{2013}. \\

  2012-- & Co-editor, McGill Working Papers in Linguistics
  (\href{http://www.mcgill.ca/mcgwpl/}{McGWPL}). \\

  2012 & Bib\TeX{} workshop (with
  \href{http://people.linguistics.mcgill.ca/~alanah.mckillen/}{Alanah
  McKillen}), McGill University, \formatdate{20}{8}{2012}. \\

  2012 & Graduate student representative for syntax job hire, McGill
  University, \shortmonthname[5]--\formatdatenoday{6}{2012}. \\

  2012 & \LaTeX{} workshop (with
  \href{http://people.linguistics.mcgill.ca/~alanah.mckillen/}{Alanah
  McKillen}), McGill University, \formatdate{2}{2}{2012}. \\

  2012 & Abstract reviewer for McCCLU (the McGill Canadian Conference for
  Linguistics Undergraduates), \formatdatenoday{1}{2012}. \\

  2011 & Student host for invited colloquium speaker Rick Nouwen, McGill
  University, \formatdatenoday{12}{2011}. \\

  2011--2012 & Co-editor, McGill Linguistics Blog
  (\href{https://blogs.mcgill.ca/mcling/}{McLing}). \\
\end{cvsection}



\section*{Languages}

\vspace{1ex}
\subsection*{Natural}

\begin{cvsection}
  English & Native \\
  Italian & Fluent \\
  German & Conversational; proficient in reading, writing \\
  French & Conversational; proficient in reading, writing \\
  Spanish & Working knowledge; research \\
  Latin & Proficient in translation; research \\
  Ancient Greek & Proficient in translation; research \\
\end{cvsection}

\vspace{-2ex}
\subsection*{Formal}

\begin{cvsection}
  Linux shell {\footnotesize (Bash)} & Advanced \\
  \LaTeX{} & Proficient \\
  HTML/CSS & Proficient \\
  Haskell & Basic \\
  Python & Basic \\
\end{cvsection}



\section*{Non-Academic Interests}

\begin{list}{}{\leftmargin=0pt}
  \item History and philosophy of math and science.
  \item Early American blues (Delta) and jazz (New Orleans).
  \item Tinkering with Linux, \LaTeX{}, and other computery things.
  \item Brewing beer, cooking.
  \item Chess, puzzles.
\end{list}



\end{document}
