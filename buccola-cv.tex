%===============================================================================
%
%  vim:     set fenc=utf8 tw=79 sw=2 sts=2 expandtab:
%  file:    buccola-cv.tex
%  author:  Brian Buccola
%
%===============================================================================



%============%
%            %
%  PREAMBLE  %
%            %
%============%

\documentclass[11pt,letterpaper]{article}



%=============================
% Packages and basic settings
%=============================

\newcommand{\name}{Brian Buccola}

\usepackage[T1]{fontenc}
\usepackage[sc,osf]{mathpazo}

\usepackage[x11names]{xcolor}
\newcommand{\linkcolor}{black}

\usepackage[margin=1in]{geometry}

\usepackage{sectsty}
  \sectionfont{\rmfamily\mdseries\bfseries\large}
  \subsectionfont{\rmfamily\mdseries\itshape\normalsize}

\usepackage[yyyymmdd]{datetime}
  \renewcommand{\dateseparator}{-}
  \newdateformat{mydate}{%
    \THEYEAR{} \shortmonthname[\THEMONTH]{} \twodigit{\THEDAY}
  }
  \mydate

\usepackage{fancyhdr}
  \pagestyle{fancy}
  \fancyhf{}
  \lhead{\textsc{\name}}
  \rhead{\textit{Curriculum vitae} / \thepage}
  \rfoot{{\footnotesize\ttfamily Last updated: \today{} \currenttime}}
  \fancypagestyle{plain}{% redefine plain page style for 1st page
    \fancyhf{}
    \renewcommand{\headrulewidth}{0pt}
    \renewcommand{\footrulewidth}{0pt}
    \rfoot{{\footnotesize\ttfamily Last updated: \today{} \currenttime}}
  }
  \thispagestyle{plain}% page style for 1st page
  \setlength{\headheight}{14pt}

\usepackage{parskip}
\setlength\parindent{0em}

\usepackage{hyperref}
  \hypersetup{%
    colorlinks  = true,
    allcolors   = \linkcolor,
    pdfauthor   = {\name},
    pdftitle    = {Curriculum Vitae - \name},
    pdfsubject  = {Curriculum Vitae},
    pdfkeywords = {\name, CV, linguistics, McGill},
    pdfpagemode = UseNone
  }



%========
% Macros
%========

\newcommand{\cvitem}[2]{%
  \begin{minipage}[t]{0.24\textwidth}
    #1 % date goes here, possibly formatted with \formatdate
  \end{minipage}
  \hfill
  \begin{minipage}[t]{0.74\textwidth}
    #2 % item details go here
  \end{minipage}
}

\newcommand{\datefill}{%
  \cvitem{\hrulefill}{}
}

\renewenvironment{itemize}{% lists without bullets
  \begin{list}{}{%
    \setlength{\itemsep}{0.5em}
    \setlength{\leftmargin}{0em}
  }
}{%
  \end{list}
}

\newcommand{\tadetails}[2]{% list of TA duties and prof
  {\footnotesize Duties included: #1. \\ Professor: #2.}
}

\newcommand{\award}[2]{%
  \begin{minipage}[t]{0.78\textwidth}
    #1        % award details
  \end{minipage}
  \hfill
  \begin{minipage}[t]{0.20\textwidth}
    \hfill #2 % dollar amount
  \end{minipage}
}



%========%
%        %
%  BODY  %
%        %
%========%

\begin{document}



%=========================================
% Title, contact info, research interests
%=========================================

\begin{minipage}[b]{0.43\textwidth}
  {\Huge\scshape \name}
\end{minipage}
\hfill
\begin{minipage}[b]{0.55\textwidth}
  \begin{flushright}
    \href{mailto:brian.buccola@mail.mcgill.ca}{\ttfamily\small
      brian.buccola@mail.mcgill.ca}                               \\
    \href{http://people.linguistics.mcgill.ca/}{\ttfamily\small
    people.linguistics.mcgill.ca/\textasciitilde{}brian.buccola}  \\
    {\footnotesize 1085 Dr--Penfield, Montr\'{e}al, Qu\'{e}bec, H2L 4H1, (312)
    945 6857}
  \end{flushright}
\end{minipage}

\vspace{-1.25em}

\hrulefill

\cvitem{\textit{Research interests}}{formal semantics, pragmatics, phonology,
computational \& mathematical linguistics, mathematical logic.}



%===========
% Main body
%===========



\section*{Education}

\cvitem{2010--present}{Ph.D. candidate, Linguistics,
  \href{http://www.mcgill.ca/}{McGill University}.}

\cvitem{2009--2010}{Qualifying year, Linguistics, McGill University.}

\cvitem{2004--2008}{B.S. Mathematics, Honors College,
  \href{http://www.luc.edu/}{Loyola University Chicago}.}

\cvitem{2004--2008}{B.A. Classics, Honors College,
  \href{http://www.luc.edu/}{Loyola University Chicago}. \\ {\footnotesize Minor
  concentrations: Latin, Ancient Greek.}}

\cvitem{1999--2004}{\href{http://www.jesuitnola.org/}{Jesuit High School New
Orleans}.}

\subsection*{Summer schools}

\cvitem{2012 \shortmonthname[6]}{\href{http://nasslli2012.com/about}{North
American Summer School of Logic, Language, and Information}.}



\section*{Refereed papers}

\cvitem{2012}{Schwarz, Bernhard, \textbf{\name}, and Michael Hamilton. ``Two
Types of Class B Numeral Modifiers: a Reply to Nouwen 2010''.  \textit{Semantics
and Pragmatics}. doi: \url{http://dx.doi.org/10.3765/sp}.}



\section*{Unpublished papers}

\cvitem{2013}{\textbf{\name}. ``Two proofs that classic OT is expressively
weaker than ordered rewrite rules''. Second evaluation paper. McGill
University.}

\cvitem{2012}{\textbf{\name}. ``Some remarks on inference patterns involving
epistemic modality''. First evaluation paper. McGill University.}



\section*{Refereed presentations}

\cvitem{\formatdate{16}{3}{2013}}{\textbf{\name}. ``A mathematical
demonstration that classic Optimality Theory is expressively weaker than
ordered rewrite rules''. Montreal--Ottawa--Toronto phonology workshop.
University of Ottawa.}

\cvitem{\formatdate{10}{3}{2012}}{\textbf{\name}. ``The nature of epistemic
implications arising from superlative quantifiers''. Toronto--Ottawa--Montreal
semantics workshop. University of Ottawa.}

\cvitem{\formatdate{9}{4}{2011}}{\textbf{\name}. ``Phrasal and clausal
comparatives: some puzzling evidence from Italian''. Toronto--Ottawa--Montreal
semantics workshop.  University of Toronto.}



\section*{Non--refereed presentations}

\cvitem{\formatdate{5}{11}{2012}}{\textbf{\name}. ``Formalization in
linguistics''. Invited lecture. Brendan Gillon's pragmatics course (LING 565).
McGill University.}

\cvitem{\formatdate{31}{7}{2012}}{\textbf{\name}. ``Some remarks on inference
patterns involving epistemic modality'', part 2. Syntax--Semantics Research
Group.  McGill University.}

\cvitem{\formatdate{11}{7}{2012}}{\textbf{\name}. ``Some remarks on inference
patterns involving epistemic modality'', part 1. Syntax--Semantics Research
Group.  McGill University.}

\cvitem{\formatdate{25}{11}{2012}}{\textbf{\name}. ``On Cummins and Katsos
(2010): Comparative and superlative quantifiers: Pragmatic effects of
comparison type''.  Syntax--Semantics Research Group. McGill University.}

\cvitem{\formatdate{22}{6}{2011}}{\textbf{\name}, Michael Hamilton, and
Bernhard Schwarz.  ``On Nouwen (2010): Two kinds of modified numerals'', part
2.  Syntax--Semantics Research Group. McGill University.}

\cvitem{\formatdate{8}{6}{2011}}{\textbf{\name}, Michael Hamilton, and Bernhard
Schwarz.  ``On Nouwen (2010): Two kinds of modified numerals'', part 1.
Syntax--Semantics Research Group. McGill University.}



\section*{Awards \& special honors}

\cvitem{2012--2013}{\award{Cremona Fellowship, Department of Linguistics, McGill
University.}{\$8,479}}

\cvitem{2012--2013}{\award{Graduate Excellence Fellowship, McGill
University.}{\$7,000}}

\cvitem{2012}{\award{NASSLLI full scholarship (tuition \& fees, accommodation,
travel stipend)}{\$410}}

\cvitem{2011--2012}{\award{Graduate Excellence Fellowship, McGill
University.}{\$6,500}}

\cvitem{2010--2011}{\award{Graduate Enrolment Recruitment Initiative, McGill
University.}{\$5,000}}

\cvitem{2010--2011}{\award{McCall MacBain Award, McGill University.}{\$10,000}}

\cvitem{2007--2008}{\award{National SMART Grant, Loyola University Chicago
(Science and Mathematics Access to Retain Talent).}{\$4,000}}

\cvitem{2004--2008}{\award{National Merit Finalist Scholarship, full tuition for
4 years, Loyola University Chicago.}{\textasciitilde\$100,000}}

\cvitem{2008 (inducted)}{\href{http://www.pbk.org/}{Phi Beta Kappa}.}

\cvitem{2004}{\href{http://www.nationalmerit.org/}{National Merit} Scholar.}

\cvitem{1999 (accepted)}{\href{http://www.mensa.org/}{Mensa}.}



\section*{Teaching}

\cvitem{2013 winter}{Phonetics (LING 330), teaching assistant, McGill
  University. \\ \tadetails{weekly lecture (2 hours/week); office hours; grade
  assignments, midterm exam, and final exam}{Morgan Sonderegger}}

\cvitem{2012 fall}{Semantics (LING 360), teaching assistant, McGill University.
  \\ \tadetails{two lectures; office hours; grade assignments, midterm exam,
  and final exam}{Brendan Gillon}}

\cvitem{2012 winter}{Phonetics (LING 330), teaching assistant, McGill
  University. \\ \tadetails{weekly lecture (2 hours/week); office hours; grade
  assignments, midterm exam, and final exam}{Heather Goad}}

\cvitem{2011 fall}{Phonology I (LING 331), teaching assistant, McGill
  University. \\ \tadetails{weekly lecture (1 hour/week); office hours; grade
  quizzes, assignments, midterm exam, and final exam}{Tobin Skinner}}

\cvitem{2011 winter}{Phonetics (LING 330), teaching assistant, McGill
  University. \\ \tadetails{weekly lecture (2 hours/week); office hours; grade
  quizzes, assignments, midterm exams, and final exam}{Sara Mackenzie}}

\cvitem{2010 fall}{Intro to Linguistics (LING 201), teaching assistant, McGill
  University. \\ \tadetails{weekly lecture (2 hours/week); office hours; grade
  quizzes, assignments, midterm exam, and final exam}{Tanya Slavin}}

\cvitem{2010 fall}{Neuroscience of Language (LING 370), grader, McGill
  University. \\ \tadetails{grade assignments, midterm exam, and final
  exam}{Isabelle Deschamps, Andrea Santi}}



\section*{Service}

\cvitem{2013 \shortmonthname[4]--\shortmonthname[5]}{Organizing committee,
Exploring the Interfaces 2: Implicatures, Alternatives, and the
Semantics--Pragmatics Interface, McGill University.}

\cvitem{\formatdate{23}{3}{2013}}{Session chair, Toronto--Ottawa--Montreal
(TOM) semantics workshop, McGill University.}

\cvitem{2013 \shortmonthname[1]}{Student host for invited colloquium speaker
Kai von Fintel, McGill University.}

\cvitem{2012--present}{Co--editor, McGill Working Papers in Linguistics
(\href{http://www.mcgill.ca/mcgwpl/}{McGWPL}).}

\cvitem{\formatdate{20}{8}{2012}}{Bib\TeX{} workshop (with Alanah McKillen),
McGill University.}

\cvitem{2012 \shortmonthname[5]--\shortmonthname[6]}{Graduate student
representative for syntax job hire, McGill University.}

\cvitem{2011--2012}{Co--editor, McGill Linguistics Blog
(\href{https://blogs.mcgill.ca/mcling/}{McLing}).}

\cvitem{\formatdate{2}{2}{2012}}{\LaTeX{} workshop (with Alanah McKillen),
McGill University.}

\cvitem{2012 \shortmonthname[1]}{Abstract reviewer for McCCLU (the McGill
Canadian Conference for Linguistics Undergraduates).}

\cvitem{2011 \shortmonthname[12]}{Student host for invited colloquium speaker
Rick Nouwen, McGill University.}



\section*{Languages}

\cvitem{\textit{Natural}}{%
  \begin{itemize}
    \item English \hfill (native)
    \item Italian \hfill (fluent)
    \item German \hfill (conversational; proficient in reading, writing)
    \item French \hfill (conversational; proficient in reading, writing)
    \item Spanish \hfill (working knowledge; research)
    \item Latin \hfill (proficient in translation; research)
    \item Ancient Greek \hfill (proficient in translation; research)
  \end{itemize}
}

\cvitem{\textit{Formal}}{%
  \begin{itemize}
    \item Linux shell (Bash)
    \item \LaTeX{}
    \item HTML
    \item CSS
    \item Haskell \hfill (beginner)
    \item Python \hfill (beginner)
  \end{itemize}
}



\section*{Non--academic interests}
\begin{description}
  \item History and philosophy of math and science.
  \item Early American blues (Delta) and jazz (New Orleans).
  \item Tinkering with Linux, \LaTeX{}, and other computery things.
  \item Brewing beer, cooking.
  \item Chess, puzzles.
\end{description}



\end{document}
